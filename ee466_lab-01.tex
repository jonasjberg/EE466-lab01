% ==============================================================================
% LAB 118
% GRUNDLÄGGNADE MÄTINSTUMENT
% --------------------------
% Last updated <2015-02-18> 
%
% Author:
% Jonas Sjöberg     <tel12jsg@student.hig.se>
% Oscar Wallberg    <emailadress@student.hig.se>
% 
% License:
% Creative Commons Attribution-NonCommercial-ShareAlike 4.0 International
% See LICENSE.md for full licensing information.
% ==============================================================================

% ==============================================================================
% INCLUDES AND CONFIGURATION
% ==============================================================================
\documentclass[draft,11pt,a4paper]{article}
\usepackage[utf8]{inputenc}
\usepackage{siunitx} % Provides the \SI{}{} and \si{} command for typesetting SI
\usepackage{amssymb}
\usepackage{amsmath}
\usepackage{amsfonts}
\usepackage{graphicx}
\usepackage{booktabs}
\usepackage{longtable} % Tables span across pages
\usepackage{microtype}
%\usepackage[swedish]{isodate}
\usepackage{gensymb}

\setlength\parindent{0pt} % Removes all indentation from paragraphs

% ==============================================================================
% DOCUMENT METADATA 
% ==============================================================================
\title{EE466 \\ Lab 118 \\ Grundläggande Mätinstrument}

\author{\\
  Jonas Sjöberg\\
  Högskolan i Gävle,\\
  Elektronikingenjörsprogrammet,\\
  \texttt{tel12jsg@student.hig.se}\\
  \\
  Oscar Wallberg\\
  Högskolan i Gävle,\\
  Dataingenjörsprogrammet,\\
  \texttt{tco13owg@student.hig.se}\\}

\date{}
% ==============================================================================
\begin{document}
% ==============================================================================
\maketitle

\begin{center}
\begin{tabular}{l r}
    % TODO
    Labb utförd: & DD Month Year \\
    Instruktör: & John Doe, John Doe
\end{tabular}
\end{center}

% ==============================================================================
% ABSTRACT
% ==============================================================================
\begin{abstract}
    Syftet med laborationen är att lära känna de vanligaste basinstrumenten i ett elektroniklaboratorium och innefattar övningar i att hantera oscilloskop, multimeter, nätaggregat, och funktionsgenerator.
\end{abstract}

\newpage

{
%\hypersetup{linkcolor=black}
\setcounter{tocdepth}{3}
\tableofcontents
}

\newpage

% ==============================================================================
% SECTION: INTRODUKTION 
% ==============================================================================
\section{Introduktion}\label{setup}
% ==============================================================================
% TODO

% ==============================================================================
% SECTION: 1.1 OSCILLOSKOPET
% ==============================================================================
\section{Oscilloskopet}\label{}
% ==============================================================================
% TODO

\subsection{Mätning av likspänning}\label{meas_dc}
% ------------------------------------------------------------------------------
% TODO

\subsection{Mätning av växelspänning}\label{meas_ac}
% ------------------------------------------------------------------------------
% TODO

% ==============================================================================
% SECTION: MULTIMETERN
% ==============================================================================
\section{Multimetern}\label{}
% ==============================================================================
% TODO

\subsection{Mätning av spänning, ström och resistans}\label{meas_multi}
% ------------------------------------------------------------------------------
% TODO

\subsection{Mätresultat}\label{TODO}
% ------------------------------------------------------------------------------
% TODO

% ==============================================================================
% SECTION: RESULTAT
% ==============================================================================
\section{Resultat}\label{setup}
% ==============================================================================
% TODO

\newpage

% ==============================================================================
% SECTION: REFERENSER
% ==============================================================================
\section{Referenser}\label{refs}
% ==============================================================================

\subsection{www}\label{interwebs}
% ------------------------------------------------------------------------------
% TODO

\subsection{Trycksaker}\label{literature} %???
% ------------------------------------------------------------------------------
% TODO

%\subsection{Källkod}\label{sourcefiles}
% ------------------------------------------------------------------------------

% ==============================================================================
\end{document}
% ==============================================================================
% ==============================================================================
% LAB 009
% TEMPERATURE SENSORS 
% ---------------------------------
% Last updated <2014-12-10> 
%
% Author:
% Jonas Sjöberg     <tel12jsg@student.hig.se>
% Esther Hedlund    <tfk13ehd@student.hig.se>
% 
% License:
% Creative Commons Attribution-NonCommercial-ShareAlike 4.0 International
% See LICENSE.md for full licensing information.
% ==============================================================================


% ==============================================================================
% INCLUDES AND CONFIGURATION
% ==============================================================================
\documentclass[draft,11pt,a4paper]{article}
\usepackage[utf8]{inputenc}
\usepackage{siunitx} % Provides the \SI{}{} and \si{} command for typesetting SI
\usepackage{amssymb}
\usepackage{amsmath}
\usepackage{amsfonts}
\usepackage{graphicx}
\usepackage{booktabs}
\usepackage{longtable} % Tables span across pages
\usepackage{microtype}
%\usepackage[swedish]{isodate}
\usepackage{gensymb}

\setlength\parindent{0pt} % Removes all indentation from paragraphs

% ==============================================================================
% DOCUMENT METADATA 
% ==============================================================================
\title{EE413 \\ Lab 009 \\ Temperature Sensors}

\author{\\
  Jonas Sjöberg\\
  Högskolan i Gävle,\\
  Elektronikingenjörsprogrammet,\\
  \texttt{tel12jsg@student.hig.se}\\
  \\
  Oscar Wallberg\\
  Högskolan i Gävle,\\
  Dataingenjörsprogrammet,\\
  \texttt{emailadress@student.hig.se}\\}

\date{}
% ==============================================================================
\begin{document}
% ==============================================================================
\maketitle

\begin{center}
\begin{tabular}{l r}
    % TODO
    Data Performed: & DD Month Year \\
    Instructor: & John Doe, John Doe
\end{tabular}
\end{center}

% ==============================================================================
% ABSTRACT
% ==============================================================================
\begin{abstract}
    % TODO: expand abstract?
Syftet med laborationen är att lära känna de vanligaste basinstrumenten i ett elektroniklaboratorium och innefattar övningar i att hantera oscilloskop, multimeter, nätaggregat, och funktionsgenerator.
\end{abstract}

\newpage

{
%\hypersetup{linkcolor=black}
\setcounter{tocdepth}{3}
\tableofcontents
}

\newpage

% ==============================================================================
% SECTION: INTRODUCTION 
% ==============================================================================
\section{Introduction}\label{setup}
% ==============================================================================
% TODO


% ==============================================================================
% SECTION: 1.1 TODO
% ==============================================================================
\section{TODO}\label{TODO}
% ==============================================================================
% TODO

\subsection{TODO}\label{TODO}
% ------------------------------------------------------------------------------
The points A and B can be expressed as in Eq~\ref{mc_1}

\begin{equation}
    R = A \times e^\frac{B}{T}
\end{equation}
\label{mc_1}

\begin{align}
    A_v     &= T_1 = 160\degree\\
            &= R_1 = 40\Omega\\
            &= T_2 = 120\degree\\
            &= R_2 = 70\Omega
            &= ln{40} = frac{ln{A}i \times B}{160}
            &= \frac{T_{out}}{V_{in}} = -\frac{R_2}{R_1}\\
            &= \frac{100k\Omega}
\end{align}


% ==============================================================================
% SECTION: Circuit prototyping setup
% ==============================================================================
\section{Circuit prototyping setup}\label{setup}
The circuit was build on a solderless breadboard, using through-hole parts.
A classic 741 op amp was used with a +/-15V power supply.
No decoupling caps was used and signal lines were not properly terminated.

For measurements the following instruments was used; HP34401A bench multimeter,
HP33120A signal generator and Agilent E3631A lab power supply.


% ==============================================================================
% SECTION: 1.2 TODO
% ==============================================================================
\section{TODO}\label{TODO}
% ==============================================================================
% TODO

\subsection{TODO}\label{TODO}
% ------------------------------------------------------------------------------
% TODO: Schematic or whatevvs

\begin{figure*}[h]
    \includegraphics{img/ad590_sch.png}
    \caption[AD590 measurement setup]{Measurement setup schematic}
    \label{ad590_sch}
\end{figure*}

The setup pictured in Fig~\ref{ad590_sch} was built on a solderless breadboard.
A thermometer was used as a reference. 

\subsection{Measurement results}\label{TODO}
% ------------------------------------------------------------------------------
% TODO


% ==============================================================================
% SECTION: RESULTS
% ==============================================================================
\section{Results}\label{setup}
% ==============================================================================
% TODO


\newpage

% ==============================================================================
% SECTION: REFERENCES
% ==============================================================================
\section{References}\label{references}
% ==============================================================================

\subsection{www}\label{literature}
% ------------------------------------------------------------------------------
% TODO

\subsection{Literature}\label{literature}
% ------------------------------------------------------------------------------
% TODO

\subsection{Source files}\label{sources}
% ------------------------------------------------------------------------------

% ==============================================================================
\end{document}
% ==============================================================================